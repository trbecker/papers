% Standard preamble %
\documentclass[a4paper]{article}
\usepackage{graphicx,hyperref,amsmath,amsfonts,bm,url,algorithmic}
\usepackage[portuges]{babel}
\usepackage[a4paper,text={16.5cm,25.2cm},centering]{geometry}
\usepackage[utf8]{inputenc}
\usepackage[retainorgcmds]{IEEEtrantools}

\setcounter{tocdepth}{2}
\setlength{\parskip}{1.2ex}
\setlength{\parindent}{0em}

\clubpenalty = 10000
\widowpenalty = 10000

\title{Problema da mochila}
\date{Novembro de 2011}
\author{Thiago Rafael Becker}

\begin{document}
\maketitle
\section*{Introdução}
{\it Definição do problema.} O problema, em sua forma original, é definido como: {\it Dado um conjunto de itens, cada um com um peso e um valor, como encher uma mochila como tal que o 	peso total não ultrapasse o limite de peso da mochila, e o valor total seja o maior possível.}

{\it Importância do problema.} Problemas similares aparecem em economia (alocação de recursos com limitação de investimento, busca de estratégias com limitação de risco), combinatória, matemática aplicada, logística ({\it bin packing}), criptografia ({\it subset sum problem}), corte de materiais com despedício mínimo.

{\it Problema de decisão}. O problema, como proposto, não serve para a análise de sua classe de complexidade, pois não é um problema de decisão. Para torná-lo um problema de decisão, estabelecemos um limite mínimo $V$ ao valor máximo dos itens na mochila, levando ao seguinte problema de decisão: {\it Dado um conjunto de $n$ itens $I$, cada um com um peso $p_i$ e um valor $v_i$ ($i \in I$), é possível encher uma mochila de forma que o peso total do itens não ultrapasse o limite de peso $P$ e o valor total seja pelo menos $V$.}

{\it Problema da mochila-0,1.} Este problema surge quando os itens são indivisíveis. De tal forma, para cada item no conjunto, a decisão é de colocá-lo na mochila (1) ou não (0).

\section*{Problema da mochila-0,1}
Com o problema de decisão definido acima, podemos definir o problema com as seguintes restrições
\begin{equation}
\label{restricao:1}
\left(\sum_{i \in I} x_i v_i\right) \ge V
\end{equation}

\begin{equation}
\label{restricao:2}
\left(\sum_{i \in I} x_i p_i\right) \le P
\end{equation}

Nas restrições acima, $x_i \in \{0,1\}$.

\subsection*{Solução por força bruta}
Como existem $n$ itens, existem $2^n$ subconjuntos de $I$. O algoritmo-solução procura por cada um dos subconjuntos pela solução ótima, o que levaria a um problema de complexidade $c_p^\le = O(2^n)$.

\subsection*{Solução em tempo pseudo-polinomial (programação dinâmica)}
Por programação dinâmica, é possível obter uma solução de complexidade $O(nP)$.

{\it Subestrutura ótima.}
Considere a carga mais valiosa $S=\{s_1, s_2, s_3, ..., s_n\}, S \subseteq I$ que pesa no máximo $P$. Se removermos o item $j$ de $S$, a carga restante deve ser a mais valiosa que pesa no máximo $P - p_j$ obtida de $n - 1$ itens de $S - j$.

Esta subestrutura ótima pode ser traduzida para a função recursiva abaixo.

\begin{equation}
\label{eq:knapsack}
B[i,w] = \left\{
\begin{IEEEeqnarraybox}[][c]{l?s}
\IEEEstrut
0 & se $i = 0$ ou $w = 0$ \\
B[i-1,w] & se $p_i > w$\\
max(B[i-1,w], B[i-1,w-p_i] + v_i)
\IEEEstrut
\end{IEEEeqnarraybox}
\right.
\end{equation}

Nesta equação, B é uma matriz de tamanho $(n + 1) \times (P + 1)$ que irá conter os resultados intermediários do cálculo.

A resposta para o problema de decisão se encontrará na célula $B[n,P]$ da matriz resultante. Se $B[n,P] \ge V$, o problema está satisfeito.

{\it Algoritmo.} O pseudo-código da solução encontra-se abaixo.
\begin{algorithmic}
\FOR{$w = 0 \to P$}
 \STATE $B[0,w] = 0$
\ENDFOR
\FOR{$i = 1 \to n$}
 \STATE $B[i,0] = 0$
\ENDFOR
\FOR{$i = 1 \to n$}
 \FOR{$w = 0 \to P$}
  \IF{$p_i \le w$}
   \IF{$v_i + B[i-1,w-p_i] > B[i-1,w]$}
    \STATE $B[i,w] = v_i + B[i-1,w-p_i]$
   \ELSE
    \STATE $B[i,w] = B[i-1,w]$
   \ENDIF
  \ELSE 
   \STATE $B[i,w] = B[i-1,w]$
  \ENDIF
 \ENDFOR
\ENDFOR

{\bf return} $B[n,P]$
\end{algorithmic}

\subsection*{Complexidade do problema}
A complexidade do algoritmo acima é $O(nP)$, na qual $n$ é a quantidade de elementos do conjunto de entrada, e $P$ é o peso máximo suportado pela mochila. É, portanto, pseudo-polinomial, pois não depende exclusivamente do tamanho do conjunto que compõe o preoblema, mas também dos parametros numéricos do problema.

\subsection*{Prova da plenitude-NP}
{\it Subset sum.} O problema {\it subset sum} será usado para provar que problema da mochila é NP-Completo. O problema é proposto da seguinte maneira: {\it Dado um conjunto de inteiros $X$, existe um subconjunto $S$ de $X$ ($X$ incluso) tal que $\left(\sum S_i\right) = c$?}

{\it Caso especial de problema da mochila.} O caso especial do problema da mochila usado para a prova será tal que $P = V$ e $(\forall i \in I)(v_i=p_i)$. Com essa equivalencia, é possível transformar a restrição \ref{restricao:2} na seguinte restrição:
\begin{equation}
\label{restricao:3}
\left(\sum_{i \in I} x_i v_i\right) \le V
\end{equation}

e igualando as restrições \ref{restricao:1} e \ref{restricao:3} pelo somatório comum, chegamos a conclusão de que a restrição do caso especial é

\begin{equation}
\label{restricao:especial}
V \le \left(\sum_{i \in I} x_i v_i\right) \le V \Rightarrow \left(\sum_{i \in I} x_i v_i\right) = V
\end{equation}

Analogamente,

\begin{equation}
\label{restricao:especial_analoga}
\left(\sum_{i \in I} x_i p_i\right) = P
\end{equation}

Este problema é equivalente ao problema de {\it subset sum}, Portanto o problema da mochila é NP-Completo.
\end{document}
