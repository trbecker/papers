\documentclass[a4paper]{article}

\usepackage{graphicx,amsmath,amsfonts,bm,url}
\usepackage[portuges]{babel}
\usepackage[a4paper,text={16.5cm,25.2cm},centering]{geometry}
\usepackage[utf8]{inputenc}

\setcounter{tocdepth}{2}
\setlength{\parskip}{1.2ex}
\setlength{\parindent}{0em}

\clubpenalty = 10000
\widowpenalty = 10000

\begin{document}
\section*{Introdução}
Git~\cite{git}\footnote{Git também é uma gíria inglesa usada para definir uma pessoa ignorante, infantil e irritante. Linus Torvalds satiriza a escolha do nome dizendo: ``Eu sou um egoísta degenerado, e vou batizar todos os meus projetos com o meu nome. Primeiro Linux, agora git.'' \cite{linus-is-a-bastard}} é o sistema de controle de versão que apresenta o maior crescimento recente graças ao suporte de ferramentas de desenvolvimento comunitário tais como github~\cite{github}. Também é uma ferramenta cuja estrutura interna é definida como um grafo dirigido acíclico. Visto isso, existiria uma categoria em um repositório? E uma categoria com todos os repositórios possíveis? ({\it Branches} seriam realmente ``endofuntores homomórficos que mapeiam subvariedades em um espaço de Hilbert?'' \cite{joke}.)

\section*{Repositórios git}
\begin{itemize}
\item Git -> scm
\item Estrutura interna, um grafo dirigido acíclico
\end{itemize}
\section*{Repositórios git como categorias}
\begin{itemize}
\item Objetos como estados do repositório, com atributos tags, files, hash
\item Morfismos como modificações do estado. Endomorfismos como nenhuma modificação, dando a noção de identidade. Todo o morfismo $A \rightarrow B$ com $A \ne B$ dá a noção de $A child\;of B$. A estrutura é similar à um conjunto parcialmente ordenado como categoria.
\end{itemize}
\section*{Transformações em repositórios como gramáticas de grafos}
\begin{itemize}
\item Transformações primárias, não são usadas diretamente, mas servem para compor as oprações sobre o repositório.
\item Composição de transformações
\item Funtores e composição de funtores
\end{itemize}
\section*{Categoria Git}
\begin{itemize}
\item Objetos como repositórios
\item Morfismos como funtores de transformação
\end{itemize}

\section*{Conclusão}

Sobre a questão levantada a respeito dos {\it branches} serem funtores homomórficos em espaços de Hilbert, esta era claramente uma brincadeira. (Será?)
\bibliography{citacoes}{}
\end{document}
