\documentclass[a4paper]{article}

\begin{document}
\section*{Introdução}
\section*{Repositórios git}
\begin{itemize}
\item Git -> scm
\item Estrutura interna, um grafo dirigido acíclico
\end{itemize}
\section*{Repositórios git como categorias}
\begin{itemize}
\item Objetos como estados do repositório, com atributos tags, files, hash
\item Morfismos como modificações do estado. Endomorfismos como nenhuma modificação, dando a noção de identidade. Todo o morfismo $A \rightarrow B$ com $A \ne B$ dá a noção de $A childof B$. A estrutura é similar à um conjunto parcialmente ordenado como categoria.
\end{itemize}
\section*{Transformações em repositórios como gramáticas de grafos}
\begin{itemize}
\item Transformações primárias, não são usadas diretamente, mas servem para compor as oprações sobre o repositório.
\item Composição de transformações
\item Funtores e composição de funtores
\end{itemize}
\section*{Categoria Git}
\begin{itemize}
\item Objetos como repositórios
\item Morfismos como funtores de transformação
\end{itemize}
\end{document}
